\documentclass[12pt,fleqn,leqno,letterpaper]{article}
\usepackage{lmodern}
\usepackage{relsize}
\usepackage[textwidth=7in,textheight=11in]{geometry}% http://ctan.org/pkg/geometry
\usepackage{amsmath, amsthm, amssymb, mathtools}
\usepackage{mathrsfs}

\title{Homework 5}
\author{Alex Day\\
	\small{Analysis of Linear Systems}
}
\date{\today}

\newtheorem{theorem}{Theorem}

\newenvironment{subproof}[1][\proofname]{%
  \renewcommand{\qedsymbol}{$\blacksquare$}%
  \begin{proof}[#1]%
}{%
  \end{proof}%
}

\begin{document}
	\maketitle

	\begin{enumerate}
		\item[1c.] A matrix $M$ is symmetric if and only if $M = M^{T}$. This implies that the matrix takes the form
			$\begin{bmatrix}
				a_{11} & a_{12} \\
				a_{21} & a_{22}
			\end{bmatrix}$
			where $a_{12} = a_{21}$. The basis for matricies of this form is:
			\begin{center}
				\boxed{
					U = \left\{ \begin{bmatrix} 1 & 0 \\ 0 & 0 \end{bmatrix}, \begin{bmatrix} 0 & 0 \\ 0 & 1 \end{bmatrix}, \begin{bmatrix} 0 & 1 \\ 1 & 0 \end{bmatrix} \right\}
				}
			\end{center}


		\item[1d.]
			\begin{align*}
				\mathcal{L}_{M}(Q) &= \begin{bmatrix} 2 & -2 \\ 6 & 5 \end{bmatrix}^{T}Q + Q \begin{bmatrix} 2 & -2 \\ 6 & 5 \end{bmatrix}\\\\
				\mathcal{L}_{M}\left(\begin{bmatrix} 1 & 0 \\ 0 & 0 \end{bmatrix}\right) &= \begin{bmatrix} 2 & -2 \\ 6 & 5 \end{bmatrix}^{T} \begin{bmatrix} 1 & 0 \\ 0 & 0 \end{bmatrix} + \begin{bmatrix} 1 & 0 \\ 0 & 0 \end{bmatrix} \begin{bmatrix} 2 & -2 \\ 6 & 5 \end{bmatrix}\\
				&= \begin{bmatrix} 4 & -2 \\ -2 & 0 \end{bmatrix} = 4 u_{1} + 0 u_{2} + (-2) u_{3}\\
				&= \begin{bmatrix} 4 & 0 & -2 \end{bmatrix}_{U}\\\\
				\mathcal{L}_{M}\left(\begin{bmatrix} 0 & 0 \\ 0 & 1 \end{bmatrix}\right) &= \begin{bmatrix} 2 & -2 \\ 6 & 5 \end{bmatrix}^{T} \begin{bmatrix} 0 & 0 \\ 0 & 1 \end{bmatrix} + \begin{bmatrix} 0 & 0 \\ 0 & 1 \end{bmatrix} \begin{bmatrix} 2 & -2 \\ 6 & 5 \end{bmatrix}\\
				&= \begin{bmatrix} 0 & 6 \\ 6 & 10 \end{bmatrix} = 0 u_{1} + 10 u_{2} + 6 u_{3}\\
				&= \begin{bmatrix} 0 & 10 & 6 \end{bmatrix}_{U}\\\\
				\mathcal{L}_{M}\left(\begin{bmatrix} 0 & 1 \\ 1 & 0 \end{bmatrix}\right) &= \begin{bmatrix} 2 & -2 \\ 6 & 5 \end{bmatrix}^{} \begin{bmatrix} 0 & 1 \\ 1 & 0 \end{bmatrix} + \begin{bmatrix} 0 & 1 \\ 1 & 0 \end{bmatrix} \begin{bmatrix} 2 & -2 \\ 6 & 5 \end{bmatrix}\\
				&= \begin{bmatrix} 12 & 7 \\ 7 & -4 \end{bmatrix} = 12 u_{1} + (-4) u_{2} + 7 u_{3}\\
				&= \begin{bmatrix} 12 & -4 & 7 \end{bmatrix}_{U}\\\\
				L_{M} &= \begin{bmatrix}
				4 & 0 & -2\\
				0 & 10 & 6\\
				12 & -4 & 7
				\end{bmatrix}
			\end{align*}

		\newpage

		\item[1f.]
			\begin{align*}
				\mathcal{L}_{M}(Q) &= \begin{bmatrix} -10 & 0 \\ 0 & -20 \end{bmatrix}\\
				&= L_{m} Q\\
				&= \begin{bmatrix}
				4 & 0 & -2\\
				0 & 10 & 6\\
				12 & -4 & 7
				\end{bmatrix} Q \\
			\end{align*}
	\end{enumerate}
\end{document}
