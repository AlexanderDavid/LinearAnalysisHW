\documentclass[12pt,fleqn,leqno,letterpaper]{article}
\usepackage{lmodern}
\usepackage{quiver}
\usepackage{relsize}
\usepackage[textwidth=7in,textheight=11in]{geometry}% http://ctan.org/pkg/geometry
\usepackage{amsmath, amsthm, amssymb, mathtools}
\usepackage{mathrsfs}

\usepackage{graphicx}
\usepackage{adjustbox}


\title{Midterm}
\author{Alex Day\\
	\small{Analysis of Linear Systems}
}
\date{\today}

\newcommand{\norm}[1]{\left\lVert#1\right\rVert}

\begin{document}
\maketitle
	\begin{enumerate}
		\item[1a.]
			\begin{align*}
			  \text{rref}(A) &= \begin{bmatrix}
								1  &  0  &  4\\
								0  &  1  &  6\\
								0  &  0  &  0\\
								0  &  0  &  0\\
								0  &  0  &  0\\
								\end{bmatrix}&&\text{calculated in matlab}\\
			  x_{1} + 4x_{3} = \bigodot_{\mathbb{R}^{5}} &\implies -4x_{3} = x_{1}\\
			  x_{2} + 6x_{3} = \bigodot_{\mathbb{R}^{5}} &\implies -6x_{3} = x_{2}\\
			  \mathcal{N}(A) &= \text{span}\left(\left\{
							    \begin{bmatrix} -4 \\ -6 \\ 1 \end{bmatrix}
								\right\}\right)\\\\
				\text{rref}(A^{T}) &= \begin{bmatrix}
													1 & 0 & 1.5 & 1 & -1.5\\
													0 & 1 & 2 & 1 & -1\\
													0 & 0 & 0 & 0 &  0\\
													\end{bmatrix}\\
			  \mathcal{R}(A^{T}) &= \text{span}\left(\left\{
							    \begin{bmatrix} -4 \\ 3 \\ 2  \end{bmatrix},
							    \begin{bmatrix} 11 \\ -8 \\ -4  \end{bmatrix}
								\right\}\right)
			\end{align*}
	  \item[1b.]
			\begin{align*}
			  U &= \left\{
					\begin{bmatrix} -4 \\ 3 \\ 2  \end{bmatrix},
					\begin{bmatrix} 11 \\ -8 \\ -4  \end{bmatrix}
					\begin{bmatrix} -4 \\ -6 \\ 1 \end{bmatrix}
					\right\}\\
			  P_{1} &=
					\begin{bmatrix} -4 & 11 & -4 \\ 3 & -8 & -6 \\ 2 & -4 & 1 \end{bmatrix}
			\end{align*}
		\item[1c.]
				\begin{align*}
					x_{1} + 1.5x_{3} + x_{4} + (-1.5)x_{5} = \bigodot &\implies x_{1} = (-1.5)x_{3} + (-1)x_{4} + 1.5x_{5} \\
					x_{2} + 2x_{3} + x_{4} + (-1)x_{5} = \bigodot &\implies x_{2} = (-2)x_{3} + (-1)x_{4} + (-1)x_{5}\\
					% \begin{bmatrix} -1.5x_{3} + -1x_{4} + 1.5x_{5}\\
					% 	-2x_{3} + -1x_{4} + -1x_{5}\\
					% 	x_{3}\\
					% 	x_{4}\\
					% 	x_{5}\end{bmatrix}\\
					\mathcal{N}(A^{T}) &= \text{span}\left(\left\{
															 \begin{bmatrix} -3\\-4\\2\\0\\0 \end{bmatrix},
															 \begin{bmatrix} -1\\-1\\0\\1\\0 \end{bmatrix},
															 \begin{bmatrix} 3\\-2\\0\\0\\2 \end{bmatrix}
															 \right\}\right)\\
					\mathcal{R}(A) &= \text{span}\left(\left\{
													 \begin{bmatrix} -8\\11\\10\\3\\1 \end{bmatrix},
													 \begin{bmatrix} 6\\-8\\-7\\-2\\-1 \end{bmatrix}
													 \right\}\right)
				\end{align*}
			\item[1d.]
				\begin{align*}
			  V &= \left\{
								\begin{bmatrix} -8\\11\\10\\3\\1 \end{bmatrix},
								\begin{bmatrix} 6\\-8\\-7\\-2\\-1 \end{bmatrix},
								\begin{bmatrix} -3\\-4\\2\\0\\0 \end{bmatrix},
								\begin{bmatrix} -1\\-1\\0\\1\\0 \end{bmatrix},
								\begin{bmatrix} 3\\-2\\0\\0\\2 \end{bmatrix}
					\right\}\\
					P_{2} &= \begin{bmatrix}
						-8 & 6 & -3 & -1 & 3\\
						11 & -8 & -4 & -1 & -2\\
						10 & -7 & 2 & 0 & 0\\
						3 & -2 & 0 & 1 & 0\\
						1 & -1 & 0 & 0 & 2
						\end{bmatrix}
				\end{align*}
			\item[1e.]
					% https://q.uiver.app/?q=WzAsNCxbMCwwLCJcXG1hdGhiYntSfV57NX0iXSxbMiwwLCJcXG1hdGhiYntSfV4zIl0sWzAsMiwiViJdLFsyLDIsIlUiXSxbMSwwLCJBIiwyXSxbMSwzLCJQXzEiXSxbMCwyLCJQXzIiLDJdXQ==
				\[\begin{tikzcd}[sep=large]
					{\mathbb{R}^{5}} && {\mathbb{R}^3} \\
					\\
					V && U
					\arrow["A"', from=1-3, to=1-1]
					\arrow["{\tilde{A}}"', from=3-3, to=3-1]
					\arrow["{P_1}", from=1-3, to=3-3]
					\arrow["{P_2}"', from=1-1, to=3-1]
				\end{tikzcd}\]
			\item[1f.]
					\begin{align*}
						\tilde{A} &= P_{2} A P_{1}^{-1}\\
						&= \begin{bmatrix}
							   43.0943 &  2.6415 &  132.2264\\
								-94.2264 & -1.3396 & -296.9434\\
								-58.6792 & -2.0189 & -182.8302\\
								-18.4151 & -0.6226 &  -57.3962\\
								-7.1698  & 0.2453  & -23.2075
	  						\end{bmatrix}
					\end{align*}
			\newpage
			\item[2a.]
				\begin{align*}
					W &= \left\{
						\begin{bmatrix} 1 & 0 \\ 0 & 0 \end{bmatrix},
						\begin{bmatrix} 0 & 1 \\ 0 & 0 \end{bmatrix},
						\begin{bmatrix} 0 & 0 \\ 1 & 0 \end{bmatrix},
						\begin{bmatrix} 0 & 0 \\ 0 & 1 \end{bmatrix}
					\right\}
				\end{align*}
			\item[2b.]
					\begin{align*}
						\mathcal{L}_{M}\left(\begin{bmatrix} 1 & 0 \\ 0 & 0 \end{bmatrix}\right) &= 2 w_{1} + (-2) w_{2} + (-2) w_{3}\\
						\mathcal{L}_{M}\left(\begin{bmatrix} 0 & 1 \\ 0 & 0 \end{bmatrix}\right) &= 2 w_{1} + 6 w_{2} + (-2) w_{4}\\
						\mathcal{L}_{M}\left(\begin{bmatrix} 0 & 0 \\ 1 & 0 \end{bmatrix}\right) &= 2 w_{1} + (-2) w_{3} + (-2) w_{4}\\
						\mathcal{L}_{M}\left(\begin{bmatrix} 0 & 0 \\ 0 & 1 \end{bmatrix}\right) &= 2 w_{2} + 2 w_{3} + 2 w_{4}\\
						L_{M} &= \begin{bmatrix} 2 & 2 & 2 & 0 \\
																		-2 & 6 & 0 & 2\\
																		-2 & 0 & -2 & 2\\
																		0 & -2 & -2 & 2\\
										\end{bmatrix}
					\end{align*}
		\item[2c.]
					\begin{align*}
						\mathcal{L}_{M}(X) &= \begin{bmatrix} 1 & 2 \\ 3 & 4 \end{bmatrix}\\
						L_{m}X_{U} &= \begin{bmatrix} 1 & 2 & 3 & 4 \end{bmatrix}_{U}^{T}\\
						X_{U} &= L_{m}^{-1}\begin{bmatrix} 1 & 2 & 3 & 4 \end{bmatrix}_{U}^{T}\\
						X_{U} &= \begin{bmatrix} 0 & -0.5 & 1 & 2.5 \end{bmatrix}^{T}_{U}\\
						X &= \begin{bmatrix} 0 & -0.5 \\ 1 & 2.5 \end{bmatrix}
					\end{align*}
		\item[2d.]
			By finding the range of the representation of $\mathcal{L}_{M}$ w.r.t. $W$ through traditional methods and then applying the trivial linear transform $T : W \rightarrow \mathbb{R}^{2}$ we can find the range of $\mathcal{L}_{M}$ w.r.t. the standard basis.
			\begin{align*}
			  L_{M} &= \begin{bmatrix}
				     1 &  -3  & -1 &  0\\
					-1 &   0  &  0 & -1\\
					-3 &   0  &  0 & -3\\
					 0  & -3  & -1 & -1\\
				      \end{bmatrix}\\
			  \text{rref}(L_{M}^{T}) &= \begin{bmatrix}
									1  &  0  &  0  &  1\\
									0  &  1  &  3  &  1\\
									0  &  0  &  0  &  0\\
									0  &  0  &  0  &  0\\
	 							  \end{bmatrix}\\
			  \mathcal{R}(L_{M}) &= \text{span}\left(\left\{
			  	\begin{bmatrix} 1 \\ -3 \\ -1 \\ 0 \end{bmatrix}, \begin{bmatrix} -1 \\ 0 \\ 0 \\ -1 \end{bmatrix}
			  \right\}\right)\\
			  \mathcal{R}(\mathcal{L}_{M}) &= \text{span}\left(\left\{
			  	\begin{bmatrix} 1 & -3 \\ -1 & 0 \end{bmatrix}, \begin{bmatrix} -1 & 0 \\ 0 & -1 \end{bmatrix}
			  \right\}\right)\\
			\end{align*}

		\end{enumerate}
\end{document}
